\documentclass[a4paper,twoside]{article}
\usepackage[utf8]{inputenc}

% \input{/home/chrberrig/Documents/LaTeX/usepackages.tex}
% \input{/home/chrberrig/Documents/LaTeX/newcommands.tex}

\usepackage{amsmath,amsfonts,amssymb,amsthm}
\newtheorem{theorem}{Theorem}
\usepackage[italicdiff]{physics}
\usepackage{graphicx}
\usepackage{float}
\usepackage{tikz}
\usepackage{tkz-euclide} %for euclidian constructions in Tikz. 
\usepackage{pgfplots}
	\setlength{\parindent}{0pt}
	\setlength{\parskip}{1ex plus 0.5ex minus 0.2ex}
\usepackage{hyperref}

%%% specific for article class %%%
\usepackage[rmargin=2.7cm,lmargin=2.7cm,bmargin=2.5cm,tmargin=2.5cm]{geometry}

% \usepackage{natbib}            % for bibtex
\usepackage[backend=biber,style=phys]{biblatex}		% for biblatex
\bibliography{/home/chrberrig/Documents/LaTeX/bib.bib}	% for biblatex

\title{the laplace transform and its inverse with exsamples.}
\usepackage[affil-it]{authblk}	% for author affiliations
\author[1]{Christian Berrig
\thanks{Electronic address: \href{mailto:chrberrig@protonmail.ch}{chrberrig@protonmail.ch}}
}
\affil[1]{Institute of Science and Environment, RUC}
\date{\today}

\begin{document}

\maketitle
% \newpage
% \tableofcontents
% \newpage



\begin{definition}[Laplace transform]
	The laplase transform of a function is defined as:
	\begin{align*}
		\mathcal{L}\qty{f(t)}(s) = \mathcal{L}\qty{f}(s) = F(s)
	\end{align*}
\begin{definition}[Laplace transform]

\begin{theorem}[Inverse laplace transform via Mellin's inverse transform/Bromwich integral]
The inverse transform of a laplace transformed function
\begin{align*}

\end{align*}
\end{theorem}

\begin{proof}
\end{proof}


 

% \bibliographystyle{plain}                                     % for bibtex
% \bibliography{/Users/chrberrig/Documents/LaTeX/bib.bib}       % for bibtex
\printbibliography      % for biblatex

\end{document}
