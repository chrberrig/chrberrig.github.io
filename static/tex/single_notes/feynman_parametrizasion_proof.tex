\documentclass[a4paper,twoside]{article}
\usepackage[utf8]{inputenc}

% \input{/home/chrberrig/Documents/LaTeX/usepackages.tex}
% \input{/home/chrberrig/Documents/LaTeX/newcommands.tex}

\usepackage{amsmath,amsfonts,amssymb,amsthm}
\newtheorem{theorem}{Theorem}
\usepackage[italicdiff]{physics}
\usepackage{graphicx}
\usepackage{float}
\usepackage{tikz}
\usepackage{tkz-euclide} %for euclidian constructions in Tikz. 
\usepackage{pgfplots}
	\setlength{\parindent}{0pt}
	\setlength{\parskip}{1ex plus 0.5ex minus 0.2ex}
\usepackage{hyperref}

%%% specific for article class %%%
\usepackage[rmargin=2.7cm,lmargin=2.7cm,bmargin=2.5cm,tmargin=2.5cm]{geometry}

% \usepackage{natbib}            % for bibtex
\usepackage[backend=biber,style=phys]{biblatex}		% for biblatex
\bibliography{/home/chrberrig/Documents/LaTeX/bib.bib}	% for biblatex

\title{Short proof of Feynman parametrizasion formula.}
\usepackage[affil-it]{authblk}	% for author affiliations
\author[1]{Christian Berrig
\thanks{Electronic address: \href{mailto:chrberrig@protonmail.ch}{chrberrig@protonmail.ch}}
}
\affil[1]{Institute of Science and Environment, RUC}
\date{\today}

\begin{document}

\maketitle
% \newpage
% \tableofcontents
% \newpage

Feynmans denominator formula , or simply feynman parametrizasion, is quite usefull and a well known result in high energy physics for evaluating certain tyles of Feynman diagrams, but i have not explicitly found (even though it of course might be out there on the internet somewhere) a detailed proof of this formula, so here is my attempt at such a detailed proof\footnote{It should be noted though, that indirect proofs are plentifull and some quite good as well. Look no further than the \href{https://en.wikipedia.org/wiki/Feynman_parametrization}{wikipedia page for Feynman parametizasion}, and you will see derivations of some simplified cases, but not the full machinery.}. This is as well an excercise for me to make this derivation, and I hope you will find it usefull:

\begin{theorem}[Feynman parametrizasion or Feynmans denominator formula]
\begin{align*}
\frac{1}{\prod_{i=1}^{n}A_{i}^{\alpha_{i}}}
= \frac{\Gamma \left( \sum_{i=1}^{n} \alpha_{i} \right)}{\prod_{i=1}^{n}\Gamma(\alpha_{i})}
 \int_{0}^{1} \dd{x_{1}}...\dd{x_{n}} 
\delta \left(1 - \sum_{i=1}^{n}x_{i}\right) 
\frac{\left( \prod_{i=1}^{n} x_{i}^{\alpha_{i} - 1} \right)}{\left( \sum_{i=1}^{n}A_{i}x_{i} \right)^{\left( \sum_{i=1}^{n} \alpha_{i} \right)}} \\
\end{align*}
\end{theorem}

\begin{proof}
First we note that the definition of the Euler-$\Gamma$ function is:
\begin{align*}
\Gamma(\alpha) = \int_{0}^{\infty} \dd{t} \left( t^{\alpha - 1} e^{-t} \right)
\end{align*}
and that for the case where $\alpha \in \mathbb{N}$, we have that:
\begin{align*}
\Gamma(\alpha) = (\alpha - 1)!
\end{align*}
It is not hard to generalize this formula a tiny amount to make another family of functions:
\begin{align*}
\Gamma_{A}(\alpha) = \int_{0}^{\infty} \dd{t} \left( t^{\alpha - 1} e^{-At} \right) = \frac{\Gamma(\alpha)}{A^{\alpha}}
\end{align*}
Now, imagine that we want the product metween $n$ of of such functions, all with different $A_{i}$ and $\alpha_{i}$, where the subscript $i$ is introduced only to distinguish between different $A_{i}$ and $\alpha_{i}$.
\begin{align*}
\prod_{i=1}^{n}\Gamma_{A_{i}}(\alpha_{i}) = \prod_{i=1}^{n} \int_{0}^{\infty} \dd{t_{i}} \left( t_{i}^{\alpha_{i} - 1} e^{-A_{i}t_{i}} \right) = \frac{\prod_{i=1}^{n}\Gamma(\alpha_{i})}{\prod_{i=1}^{n}A_{i}^{\alpha_{i}}}
\end{align*}
It is important to note here as well that this identity is not necessarily true for infinite values of $n$, thus we restrict ourselves to the case of finite $n$.
To proceed from here, let the sum of all the integraiton variable be $s$, such that:
\begin{align*}
s = \sum_{i=1}^{n}t_{i}
\end{align*}
To clarify, the ranges of each of the $t_{i} = [0, \infty]$, $\forall i \in \{1, 2, ... , n\}$, and thus we have that $s = [0, \infty]$
we can expand the integral part of the previous equation, by using a dirac-$\delta$ function such that:
\begin{align*}
\prod_{i=1}^{n}\Gamma_{A_{i}}(\alpha_{i}) 
= \int_{0}^{\infty} \dd{s} \left( \prod_{i=1}^{n} \int_{0}^{\infty} \dd{t_{i}} \left( t_{i}^{\alpha_{i} - 1} e^{-A_{i}t_{i}} \right) \right) \delta \left(s - \sum_{i=1}^{n}t_{i}\right) 
= \frac{\prod_{i=1}^{n}\Gamma(\alpha_{i})}{\prod_{i=1}^{n}A_{i}^{\alpha_{i}}}
\end{align*}
From here, new normalized variables $x_{i}$ are introdused
\begin{align*}
x_{i} = \frac{t_{i}}{s} \quad , \quad x_{i} \in [0, 1]
\end{align*}
We can now make the following substitution\footnote{I will ask you to forgive my notation; when I write $\int_{0}^{1} \dd{x_{1}}...\dd{x_{n}}$, I implicitly refer to the nested integrals$\int_{0}^{1} \dd{x_{1}}...\int_{0}^{1}\dd{x_{n}}$ it is simply a notation that in the attempt of minimizing the clutter in writing, abuse the fact that $x_{i} \in [0, 1] \quad,\forall i \in \{1, ... , n\} $}
\begin{align*}
\prod_{i=1}^{n}\Gamma_{A_{i}}(\alpha_{i}) 
= \int_{0}^{\infty} \dd{s} \left( \prod_{i=1}^{n} \int_{0}^{1} \dd{x_{i}} s \left( (s x_{i})^{\alpha_{i} - 1} e^{-A_{i}s x_{i}} \right) \right) \delta \left(s \left(1 - \sum_{i=1}^{n}x_{i} \right) \right) 
\end{align*}
and using the deltafunction identity that:
\begin{align*}
\delta \left(s \left(1 - \sum_{i=1}^{n}x_{i} \right) \right) = \frac{1}{s}\delta \left(1 - \sum_{i=1}^{n}x_{i} \right)
\end{align*}
we can write the above equation as:
\begin{align*}
\prod_{i=1}^{n}\Gamma_{A_{i}}(\alpha_{i}) 
=& \int_{0}^{\infty} \dd{s} \left( \prod_{i=1}^{n} \int_{0}^{1} \dd{x_{i}} s \left( (s x_{i})^{\alpha_{i} - 1} e^{-A_{i}s x_{i}} \right) \right) \frac{1}{s} \delta \left(1 - \sum_{i=1}^{n}x_{i} \right) \\
=& \int_{0}^{\infty} \dd{s} \left( \int_{0}^{1} \dd{x_{1}}...\dd{x_{n}} s^{n} \left( \prod_{i=1}^{n} s^{\alpha_{i} - 1} x_{i}^{\alpha_{i} - 1} e^{-A_{i}s x_{i}} \right) \right) \frac{1}{s} \delta \left(1 - \sum_{i=1}^{n}x_{i}\right) \\
=&  \int_{0}^{1} \dd{x_{1}}...\dd{x_{n}} \left( \prod_{i=1}^{n} x_{i}^{\alpha_{i} - 1} \right) \delta \left(1 - \sum_{i=1}^{n}x_{i}\right) 
\int_{0}^{\infty} \dd{s} \left( \frac{1}{s} s^{n} s^{\sum_{i=1}^{n}(\alpha_{i} - 1)} e^{-s\sum_{i=1}^{n}(A_{i} x_{i})} \right) \\
=&  \int_{0}^{1} \dd{x_{1}}...\dd{x_{n}} \left( \prod_{i=1}^{n} x_{i}^{\alpha_{i} - 1} \right) \delta \left(1 - \sum_{i=1}^{n}x_{i}\right) 
\int_{0}^{\infty} \dd{s} \left( s^{\sum_{i=1}^{n}(\alpha_{i}) - 1} e^{-s\sum_{i=1}^{n}(A_{i} x_{i})} \right)
\end{align*}
and noting that the $s$-integral has the exact form of the $\Gamma_{A}(\alpha)$ function, we can make the identification, 
\begin{align*}
\int_{0}^{\infty} \dd{s} \left( s^{\sum_{i=1}^{n}(\alpha_{i}) - 1} e^{-s\sum_{i=1}^{n}(A_{i} x_{i})} \right) 
= \Gamma_{\sum_{i=1}^{n}A_{i}x_{i}} \left( \sum_{i=1}^{n} \alpha_{i} \right) 
= \frac{\Gamma \left( \sum_{i=1}^{n} \alpha_{i} \right)}{\left( \sum_{i=1}^{n}A_{i}x_{i} \right)^{\left( \sum_{i=1}^{n} \alpha_{i} \right)}} 
\end{align*}
and thus:
\begin{align*}
\prod_{i=1}^{n}\Gamma_{A_{i}}(\alpha_{i}) 
=&  \int_{0}^{1} \dd{x_{1}}...\dd{x_{n}} \left( \prod_{i=1}^{n} x_{i}^{\alpha_{i} - 1} \right) \delta \left(1 - \sum_{i=1}^{n}x_{i}\right) 
\int_{0}^{\infty} \dd{s} \left( s^{\sum_{i=1}^{n}(\alpha_{i}) - 1} e^{-s\sum_{i=1}^{n}(A_{i} x_{i})} \right) \\
=&  \int_{0}^{1} \dd{x_{1}}...\dd{x_{n}} \left( \prod_{i=1}^{n} x_{i}^{\alpha_{i} - 1} \right) \delta \left(1 - \sum_{i=1}^{n}x_{i}\right) 
\frac{\Gamma \left( \sum_{i=1}^{n} \alpha_{i} \right)}{\left( \sum_{i=1}^{n}A_{i}x_{i} \right)^{\left( \sum_{i=1}^{n} \alpha_{i} \right)}} \\
=& \Gamma \left( \sum_{i=1}^{n} \alpha_{i} \right)
 \int_{0}^{1} \dd{x_{1}}...\dd{x_{n}} 
\delta \left(1 - \sum_{i=1}^{n}x_{i}\right) 
\frac{\left( \prod_{i=1}^{n} x_{i}^{\alpha_{i} - 1} \right)}{\left( \sum_{i=1}^{n}A_{i}x_{i} \right)^{\left( \sum_{i=1}^{n} \alpha_{i} \right)}} \\
\end{align*}
since the $\Gamma \left( \sum_{i=1}^{n} \alpha_{i} \right)$ factor is independent of the $x_{i}$
Now, using what we have found, remember what the goal was; to find a integral-parametrized form of $\frac{1}{\prod_{i=1}^{n}A_{i}^{\alpha_{i}}}$, we can also remember that the initial representation we found for the $\Gamma_{A}(\alpha)$ function, we have actually (almost) arrivedat the end result:
\begin{align*}
\prod_{i=1}^{n}\Gamma_{A_{i}}(\alpha_{i}) 
= \frac{\prod_{i=1}^{n}\Gamma(\alpha_{i})}{\prod_{i=1}^{n}A_{i}^{\alpha_{i}}}
= \Gamma \left( \sum_{i=1}^{n} \alpha_{i} \right)
 \int_{0}^{1} \dd{x_{1}}...\dd{x_{n}} 
\delta \left(1 - \sum_{i=1}^{n}x_{i}\right) 
\frac{\left( \prod_{i=1}^{n} x_{i}^{\alpha_{i} - 1} \right)}{\left( \sum_{i=1}^{n}A_{i}x_{i} \right)^{\left( \sum_{i=1}^{n} \alpha_{i} \right)}} \\
\end{align*}
whereby we can conclude with the formula:
\begin{align*}
\frac{1}{\prod_{i=1}^{n}A_{i}^{\alpha_{i}}}
= \frac{\Gamma \left( \sum_{i=1}^{n} \alpha_{i} \right)}{\prod_{i=1}^{n}\Gamma(\alpha_{i})}
 \int_{0}^{1} \dd{x_{1}}...\dd{x_{n}} 
\delta \left(1 - \sum_{i=1}^{n}x_{i}\right) 
\frac{\left( \prod_{i=1}^{n} x_{i}^{\alpha_{i} - 1} \right)}{\left( \sum_{i=1}^{n}A_{i}x_{i} \right)^{\left( \sum_{i=1}^{n} \alpha_{i} \right)}} \\
\end{align*}
\end{proof}


 

% \bibliographystyle{plain}                                     % for bibtex
% \bibliography{/Users/chrberrig/Documents/LaTeX/bib.bib}       % for bibtex
\printbibliography      % for biblatex

\end{document}
