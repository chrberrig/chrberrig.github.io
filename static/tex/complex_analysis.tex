\documentclass[a4paper,twoside]{article}
\usepackage[utf8]{inputenc}

\usepackage{amsmath,amsfonts,amssymb,amsthm}
\newtheorem{theorem}{Theorem}
\newtheorem{corollary}{Corollary}[theorem]
% \theoremstyle{definition}
\newtheorem{definition}{Definition} %[section]

\usepackage[italicdiff]{physics}
\usepackage{graphicx}
\usepackage{float}
\usepackage{tikz}
\usepackage{tkz-euclide} %for euclidian constructions in Tikz. 
\usepackage{pgfplots}
	\setlength{\parindent}{0pt}
	\setlength{\parskip}{1ex plus 0.5ex minus 0.2ex}
\usepackage{hyperref}

%%% specific for article class %%%
\usepackage[rmargin=2.7cm,lmargin=2.7cm,bmargin=2.5cm,tmargin=2.5cm]{geometry}

% \usepackage{natbib}            % for bibtex
\usepackage[backend=biber,style=phys]{biblatex}		% for biblatex
\bibliography{/home/chrberrig/Documents/LaTeX/bib.bib}	% for biblatex

\title{Complex analysis (basics)}
\usepackage[affil-it]{authblk}	% for author affiliations
\author[1]{Christian Berrig
\thanks{Electronic address: \href{mailto:chrberrig@protonmail.ch}{chrberrig@protonmail.ch}}
}
\affil[1]{Institute of Science and Environment, RUC}
\date{\today}

\begin{document}

\maketitle
% \newpage
% \tableofcontents
% \newpage

\begin{definition}[Analytic/holomorphic function] 
	For a conplex variable $z$, we say that the complex function $f(z)$ is analytic or holomorphic at $z$ (and in a neighgbourhood around $z$), iff:
	\begin{align*}
		f'(z) = \lim_{z' \rightarrow z} \dfrac{f(z) - f{z'}}{z - z'}
	\end{align*}
	exists and is well defined as $z' \rightarrow z$ from any direction in the complex plane \\
	A function analytic on all points in the complex plane is also said to be entire.
\end{definition}

\begin{corollary}[Cauchy–Riemann equations]
	For any analytic function, $f(z) = u + iv$, where $z = x + iy$, the following holds:
	\begin{align*}
		\partial{u}{x} &= \partial{v}{y} \\
		\partial{u}{y} &= -\partial{v}{x}
	\end{align*}
\end{corollary}
\begin{proof}
	This is prooved by the choise of direction of integrability: take respectively the real and imaginary axis as directions for differentiation to find that:
	
\end{proof}

% \begin{theorem}[Cauchy's integral theorem]
% 
% \begin{align*}
% \end{align*}
% \end{theorem}
% 
% \begin{proof}
% \end{proof}
% 

 

% \bibliographystyle{plain}                                     % for bibtex
% \bibliography{/Users/chrberrig/Documents/LaTeX/bib.bib}       % for bibtex
\printbibliography      % for biblatex

\end{document}
